\documentclass[xcolor=dvipsnames, 14pt]{beamer}
%\documentclass[xcolor=dvipsnames, bigger, aspectratio=169]{beamer}

\definecolor{Saffron}{HTML}{F4C430}
\usecolortheme[named=Saffron]{structure}

\mode<presentation> {
	\usetheme[height=2em]{Rochester}
	\setbeamercovered{transparent}
}

\setbeamertemplate{caption}{\insertcaption\par}
\setbeamertemplate{navigation symbols}{}%remove navigation symbols

\setbeamercolor{frametitle}{fg=black}
\setbeamercolor{title}{fg=black}
\setbeamercolor{navigation symbols dimmed}{fg=black!10}
\setbeamercolor{navigation symbols}{fg=black!30}
\setbeamercolor{section number projected}{fg=black}
\setbeamercolor{item projected}{fg=black}


\usepackage[utf8x]{inputenc}
\usepackage[resetfonts]{cmap}
\usepackage{lmodern}
\usepackage[czech]{babel}
\usepackage[T1]{fontenc}

\usepackage{graphicx}
\usepackage{amsmath}
\usepackage{amssymb}
\usepackage{listings}
\usepackage{microtype}
\usepackage{tikz}

\usepackage{hyperref}
\hypersetup{unicode=true}

% ----- macros -----
\newcommand{\imageW}[1]{%
  \makebox[\textwidth][c]{\includegraphics[width=1.12\textwidth]{img/#1}}}
\newcommand{\imageH}[1]{%
  \makebox[\textwidth][c]{\includegraphics[height=0.97\textheight]{img/#1}}}

% ---- info -----
\title{Adaptabilní programování}
\author{Jaroslav~Čechák \and Tomáš~Effenberger \and  Jiří~Mauritz}
\institute{Fakulta informatiky, Masarykova univerzita}
\date{\today}

\begin{document}

\begin{frame}
\titlepage
\end{frame}

\begin{frame}
\frametitle{Cíl}
\begin{itemize}
\item aplikace pro efektivní učení programování
\item učení užitečných dovedností
\item zábavné úlohy optimální obtížnosti
\item stav \emph{flow} $\rightarrow$ maximalizace učení
\end{itemize}
\end{frame}

\begin{frame}
\frametitle{Flow}
\begin{figure}[h]
  \centering
  \begin{tikzpicture}[font=\sffamily,xscale=5, yscale=5]
  \large
  %\draw [lightgray, fill=gray] (0,0) -- (0.1,0) -- (1,0.8) -- (0.8,1) -- (0,0.1) -- (0,0);
  \draw (0.1,0) -- (1,0.8);
  \draw (0,0.1) -- (0.8,1);
  \draw [thick, <->] (0,1) node [left] {obtížnost} -- (0,0) -- (1,0) node [below right] {dovednost};
  \node at (0.27,0.82) {\emph{frustrace}};
  \node at (0.6,0.6) {\emph{flow}};
  \node at (0.7,0.2) {\emph{nuda}};
  \end{tikzpicture}
  %\caption{Relationship between challenge and skill.}
\end{figure}
\end{frame}
% NOTE: jak to delame: Elo model

\begin{frame}
\frametitle{Druhý prototyp}
% NOTE: prvni prototyp byl takovy proof of concept
% - ted: 2. prototyp - uz by mel byt opravdu pouzitelny, chceme publikovat do par tydny
\imageW{practice.png}
\end{frame}

\begin{frame}
\frametitle{Druhý prototyp}
\imageW{task-completion-modal.png}
\end{frame}

\begin{frame}
\frametitle{Druhý prototyp}
\imageH{task-example.png}
% NOTE: ulohy muzou obsahovat klice, jamy, barvy
\end{frame}

\begin{frame}
\frametitle{Druhý prototyp}
\imageW{statistiky.png}
\end{frame}

\begin{frame}
\frametitle{Druhý prototyp}
\imageW{bloky.png}
\end{frame}

\begin{frame}
\frametitle{Druhý prototyp}
\imageW{solved-tasks.png}
\end{frame}


\begin{frame}
\frametitle{Co jsme dělali} % zmeny od 1. prototypu tj. od ledna
\begin{itemize}
\item konečné procvičovací relace
\item motivace a interpretace dovednosti % kredity, kupovani bloku
\item nové úlohy
\item uživatelské rozhraní
\item uživatelské statistiky
\item admin statistiky
\item sbírání a export dat
\item modely pro koncepty a instrukce
\item i18n and l10n
\end{itemize}
\end{frame}

\begin{frame}
\frametitle{Na čem pracujeme}
\begin{itemize}
\item příjemnější uživatelské rozhraní
\item vizualizace instrukcí
\item vylepšení exportu dat
\item nasazení a otestování druhého prototypu
\end{itemize}
% NOTE: a nejake technicke veci, napr. doreseni autentizace uzivatelu
\end{frame}

\begin{frame}
\frametitle{Dlouhodobé plány}
\begin{itemize}
\item analýza modelů a doporučovacích technik
\item téma, příběh, grafika
\item intuitivní uživatelské rozhraní
\item ještě více úloh
\end{itemize}
\end{frame}

\begin{frame}
\frametitle{Shrnutí}
\begin{itemize}
\item systém pro adaptabilní výuku programování % snaha o flow
\item úlohy v bludišti, programování pomocí bloků
\item první prototyp: \small{\url{http://flocs.thran.cz}}\normalsize
\item druhý prototyp: začátek července
\item repozitář: \small{https://github.com/adaptive-learning/flocs}
\end{itemize}
\end{frame}

\end{document}
