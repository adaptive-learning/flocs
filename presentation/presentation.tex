\documentclass[xcolor=dvipsnames, 14pt]{beamer}
%\documentclass[xcolor=dvipsnames, bigger, aspectratio=169]{beamer}

\definecolor{Saffron}{HTML}{F4C430}
\usecolortheme[named=Saffron]{structure}

\mode<presentation> {
	\usetheme[height=2em]{Rochester}
	\setbeamercovered{transparent}
}

\setbeamertemplate{caption}{\raggedright Obr\'azek: %\arabic{figure}.
\insertcaption\par}
\setbeamertemplate{navigation symbols}{}%remove navigation symbols

\setbeamercolor{frametitle}{fg=black}
\setbeamercolor{title}{fg=black}
\setbeamercolor{navigation symbols dimmed}{fg=black!10}
\setbeamercolor{navigation symbols}{fg=black!30}
\setbeamercolor{section number projected}{fg=black}
\setbeamercolor{item projected}{fg=black}


\usepackage[utf8x]{inputenc}
\usepackage[resetfonts]{cmap}
\usepackage{lmodern}
\usepackage[english]{babel}
\usepackage[T1]{fontenc}

\usepackage{graphicx}
\usepackage{amsmath}
\usepackage{amssymb}
\usepackage{listings}
\usepackage{microtype}

\usepackage{hyperref}
\hypersetup{unicode=true}

\title{Adaptive Programming}
\author{Jaroslav~Čechák \and Tomáš~Effenberger \and  Jiří~Mauritz \and Jakub Peschel}
\institute{Faculty of Informatics, Masaryk University}
\date{\today}

\begin{document}

\begin{frame}
\titlepage
\end{frame}

\begin{frame}
\frametitle{Goal}
TODO (both long term and short term goal = within RecSys course)
\end{frame}

\begin{frame}
\frametitle{Flow}
TODO (flow plot to remind the concept of flow as we will use this term later and is essential to the domain of educational recommendations)
\end{frame}

\begin{frame}
\frametitle{Existing projects and research}
TODO (this can be similar to what we prepared for the ALG seminar, but should be brief)
\end{frame}

\begin{frame}
\frametitle{Screenshots}
TODO: some screenshots so that everyone has an idea about what we are doing
(+ better slide name or just large screenshots without frame title)
\end{frame}

\begin{frame}
\frametitle{Models}
TODO: model of student skills and task difficulty
\end{frame}

\begin{frame}
\frametitle{Tasks}
TODO: details about tasks, incl.generating initial difficulty
\end{frame}

\begin{frame}
\frametitle{Prediction}
TODO: how we predict flow
\end{frame}

\begin{frame}
\frametitle{Task selection}
TODO: how we select the best task
\end{frame}

\begin{frame}
\frametitle{Parameters update}
TODO: 2 phases: (1) update based on report, (2) learning shift
\end{frame}

\begin{frame}
\frametitle{Simulations}
TODO: plots of normal, stupid, genius students...
\end{frame}

\begin{frame}
\frametitle{Implementation}
TODO: implementation details (overview + interesting parts)
\end{frame}

\begin{frame}
\frametitle{Evaluation}
\begin{itemize}
\item not enough data
\end{itemize}
TODO: (at least we can show some basic statistics and how the task difficulties changed in time)
\end{frame}

\begin{frame}
\frametitle{Experience report}
TODO: present experience reports from users
\end{frame}

\begin{frame}
\frametitle{Summary}
\begin{itemize}
\item system for adpative learning of programming
\item Blockly language, tasks in a maze
\item concept of flow, Elo model
\item first prototype: link
\item source codes: link
\end{itemize}
\end{frame}

\end{document}
